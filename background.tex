\section{Introduction}
\label{s:intro}
Locating a radio-frequency transmitter, or \emph{source localization}, is a vital step in many applications. Source localization is often performed using measurements of the transmitted signal obtained by several nearby receivers. Specifically, measurements of the transmitted signal at two distinct receivers allow one to compute the time difference of arrival (TDOA) and frequency difference of arrival (FDOA) between those receivers. With estimates of TDOA or FDOA measurements, one can compute various other quantities describing the location of the transmitter, including the angle of arrival (AOA) / direction of arrival (DOA), the range to the receiver, and thus the location of the transmitter in the global coordinate system (geolocation). If information about the source is known {\em a priori}, such as altitude (ALT), it is typically possible to estimate receiver location with the use of fewer measurements.\\

Source localization using only TDOA measurements is a well-understood problem and many algorithms have been developed for its solution. Common approaches include linearization of the system or a multidimensional search~\cite{Torrieri1984}. Methods for managing data to deal with noise, including divide and conquer (DAC), the RANdom SAmpling Consensus method (RANSAC), and projection to the feasible set of TDOA measurements have been proposed ~\cite{Cameron,Abel1990,Li2009,Compagnoni2017}. Geometrically, each TDOA measurement restricts the potential transmitter location to a hyperboloid. Thus, if several measurements are obtained, locating the emitter requires finding the intersection of several hyperboloids. Simple geometric relationships between the TDOA measurements and the known receiver positions allow the DOA to be computed with a single antenna array~\cite{Benesty2008}. It follows that with multiple antenna arrays, the source can also be located via triangulation. \\

The equations relating FDOA measurements to receiver positions are more complicated than the corresponding TDOA equations. The FDOA model is nonlinear and depends on the receiver velocities, so source localization with FDOA measurements is more complicated than geolocation using TDOA measurements. Additionally, since FDOA measurements quantify the Doppler effect between receivers, it is essential that each receiver has a different velocity. This makes FDOA approximation with an antenna array impossible. \\

While the FDOA measurements are often used as an additional constraint to the TDOA geolocation systems (TDOA/FDOA localization)~\cite{Ho1997}, only a few algorithms have been developed using FDOA alone~\cite{Cameron,Jinzhou2012}. Some limitations of these algorithms are the high cost of computation that comes from nonlinear solver methods. There are, however, some cases where it is desirable to solve for the emitter location using FDOA only. For instance, in the case of a narrowband signal with a long pulse duration, Doppler resolution is finer than the range resolution so that it is difficult to measure the TDOA accurately~\cite{Cheney2009,Mason2005,Jinzhou2012}. \\

When the distance between the receivers and the transmitter is much greater than the distance between the receivers it is common to simplify the wave propagation model and assume that wave curvature is negligible in the region of the receivers. This assumption is commonly referred to as the far-field assumption~\cite{Cheney2009}. In this paper we present how this assumption can reduce the computation of DOA to the solution of a linear system. \\

While DOA estimation is typically performed with a TDOA-based strategy, our approach is able to utilize TDOA or FDOA measurements, or both, by capitalizing on the simplified geometry of the source-localization problem under the far-field assumption. One scenario where this method might be useful is in the calculation of DOA of a narrowband emitter using several receivers. The main benefit of this method is its computational efficiency, as it simplifies the calculation of DOA to solving a linear system of equations.  With several DOA calculations, triangulation can be used to determine location of the source. In section~\ref{s:FDOA}, we develop a far-field model for the FDOA measurements and discuss a technique for determining the signal direction of arrival. In section~\ref{s:TDOA}, we develop a similar far-field approximation for the TDOA model and present the analogous DOA technique. Finally, we summarize the method with some numerical results in section~\ref{s:numerics} and concluding remarks in section~\ref{s:conclusion}.
