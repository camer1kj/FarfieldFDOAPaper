\section{Introduction}
\label{s:intro}
Locating a radio-frequency transmitter is a vital step in many applications. Source localization is often performed using measurements of the transmitted signal obtained by several nearby receivers. Specific observable quantities include the time difference of arrival (TDOA) and frequency difference of arrival (FDOA) between receivers. When the distance between the receivers and the transmitter is much greater than the distance between the receivers (far-field case~\cite{Cheney2009}), the angle of arrival (AOA) can be found using the geometric relationship with the TDOA and distance between receivers~\cite{Benesty2008}. Additionally, the problem can be simplified if information about the transmitter is known {\em a priori}, such as altitude (ALT). This paper introduces a novel approach to calculating the AOA by capitalizing on the interesting geometry of the source-localization problem. This method allows for the calculation of AOA using either TDOA or FDOA measurements and simplifies the problem to the solution of a linear system of equations. \\

The FDOA equations are nonlinear and thus have a more complicated geometry than geolocation with TDOA measurements only. While the FDOA measurements are often used as an additional constraint to the TDOA geolocation systems (TDOA/FDOA localization)~\cite{Ho1997}, only a few algorithms have been developed using FDOA alone (~\cite{Cameron,Jinzhou2012}). There are some special cases where it is desirable to solve for the emitter location using FDOA only. For instance in the case of a narrowband signal with a long pulse duration, the Doppler resolution is higher than the range resolution and it can be difficult to measure the TDOA accurately~\cite{Cheney2009,Mason2005,Jinzhou2012}. \\

As mentioned above, the angle of arrival (AOA), sometimes referred to as the direction of arrival (DOA), is usually calculated using basic trigonometry and TDOA measurements. A nice overview of this relationship is covered in~\cite{Benesty2008}. For a source in the near-field, both the ranges and AOA can be found using this method with a single array of sensors. In the far-field case, only the AOA can be calculated, so more than one set of measurements are needed for source localization~\cite{Benesty2008}. In this paper, we present an alternative method for calculation of AOA using FDOA measurements in the far-field. \\

We derive the FDOA far-field approximation and its use for determining direction of arrival in section \ref{s:FDOA}. A proposed source-localization algorithm is presented in section \ref{s:algorithm}.
