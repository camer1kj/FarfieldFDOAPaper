\section{Introduction}
\label{s:intro}
% Notes:
% Maybe we should it include a few citations about applications which require locating an RF transmitter. We can just steal these from another TDOA / FDOA paper
Locating a radio-frequency transmitter, or \emph{source localization}, is a vital step in many applications. Source localization is often performed using measurements of the transmitted signal obtained by several nearby receivers. Specifically, measurements of the transmitted signal at two distinct receivers allow one to compute the time difference of arrival (TDOA) and frequency difference of arrival (FDOA) between those receivers. With estimates of TDOA or FDOA measurements, one can compute various other quantities describing the location of the transmitter, including the angle of arrival (AOA) / direction of arrival (DOA), the range to the receiver, and thus the location of the transmitter in the global coordinate system (geolocation). If information about the source is known {\em a priori}, such as altitude (ALT), it is typically possible to estimate receiver location with the use of fewer measurements.\\

Source localization using only TDOA measurements is a well-understood problem and many algorithms have been developed for its solution. Common approaches include linearization of the system or a multidimensional search~\cite{Torrieri1984}. Methods for managing data to deal with noise, including divide and conquer (DAC), the RANdom SAmpling Consensus method (RANSAC), and projection to the feasible set of TDOA measurements have been proposed ~\cite{Cameron,Abel1990,Li2009,Compagnoni2017}. Geometrically, each TDOA measurement restricts the potential transmitter location to a hyperboloid. Thus if several measurements are obtained, locating the emitter requires finding the intersection of several hyperboloids. \\

For a nearby source, simple geometric relationships between the TDOA measurements and the known receiver positions allow the DOA to be computed with an array of sensors~\cite{Benesty2008}. When the distance between the receivers and the transmitter is much greater than the distance between the receivers it is common to simplify the wave propagation model and assume that wave curvature is negligible in the region of the receivers. This assumption is commonly referred to as the far-field assumption~\cite{Cheney2009}. This simplification reduces the computation of the DOA to the solution of a linear system. \\

% Can we say anything about the limitations of these algorithms? It would give more motivation for us to develop our method. E.g. relatively high cost of computation, numerical instability, etc.
The equations relating the FDOA measurements to the receiver positions are more complicated than those used in the TDOA case. The FDOA model is nonlinear and depends on the receiver velocities, so source localization with FDOA measurements is more complicated than geolocation using TDOA measurements. While the FDOA measurements are often used as an additional constraint to the TDOA geolocation systems (TDOA/FDOA localization)~\cite{Ho1997}, only a few algorithms have been developed using FDOA alone~\cite{Cameron,Jinzhou2012}. Some limitations of these algorithms are the high cost of computation that comes from nonlinear solver methods. There are, however, some cases where it is desirable to solve for the emitter location using FDOA only. For instance, in the case of a narrowband signal with a long pulse duration, Doppler resolution is finer than the range resolution so that it is difficult to measure the TDOA accurately~\cite{Cheney2009,Mason2005,Jinzhou2012}. \\

While DOA estimation is typically performed using TDOA measurements, this paper introduces a novel approach to calculating the DOA using either TDOA or FDOA measurements, or both, by capitalizing on the simplified geometry of the source-localization problem under the far-field assumption. The main benefit of this method is its computational efficiency, as it simplifies the calculation of DOA to the solution of a linear system of equations. With several DOA calculations, triangulation can be used to determine location of the source. In section~\ref{s:FDOA}, we develop a far-field model for the FDOA measurements and discuss a technique for determining the signal direction of arrival. In section~\ref{s:TDOA}, we develop a similar far-field approximation for the TDOA model and present the analogous DOA technique. Finally, we summarize the method with some numerical results in section~\ref{s:numerics} and include a discussion of future directions for research in section~\ref{s:conclusion}.
