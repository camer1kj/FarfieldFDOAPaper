\section{Far-field approximation for FDOA}
\label{s:FDOA}
Consider a stationary transmitter located at $\mbf{x}$. Additionally, there are $N$ receivers located at $\mbf{x}_1, ...,\mbf{x}_N$ with velocity $\mbf{v}_1, ...,\mbf{v}_N$. The frequency shift of the signal between the emitter and receiver $i$ is,
\begin{align}
  \label{eq:fshift}
  d_i =  \mbf{v}_i^T\cdot\frac{\mbf{x}_i-\mbf{x}}{\|\mbf{x}_i-\mbf{x}\|}.
\end{align}
We now derive a far-field approximation for equation \ref{eq:fshift}.

Assume without loss of generality that the receivers are centered around the origin. We consider the far-field case, where the distance between receivers is much smaller than the distance to the emitter, i.e. $\|\mathbf{x}\|>>\|\mathbf{x}_i\|, \; \forall \; i$. We approximate $\|\mathbf{x-x_i}\|$ by expanding:
\begin{align*}
  \|\mathbf{x-x_i}\| \approx \|\mathbf{x}\|-\hat{\mathbf{x}}^T\mathbf{x_i}+\mathcal{O}(\dfrac{\|\mathbf{x}_i\|}{\|\mathbf{x}\|}).
\end{align*}
Truncating after the first term above allows for simplification of the factor (in eq. \ref{eq:fshift}):
\begin{align*}
  \frac{\mbf{x}_i-\mbf{x}}{\|\mbf{x}_i-\mbf{x}\|} \approx \frac{\mbf{x}_i}{\|\mbf{x}\|}-\frac{\mbf{x}}{\|\mbf{x}\|}.
\end{align*}
Additionally, the far-field assumption implies that the first term will have small magnitude. Thus, $\dfrac{\mathbf{x_i-x}}{\|\mathbf{x_i-x}\|}$ is simplified to $\dfrac{-\mathbf{x}}{\|\mathbf{x}\|}$.
Equation \ref{eq:fshift} becomes:
\begin{align}
  d_i \approx  -\mbf{v}_i^T\cdot\hat{\mbf{x}},
\end{align}
where $\hat{\mbf{x}} = \frac{\mbf{x}}{\|\mbf{x}\|}$, is a unit vector from the centroid of the receivers. The entire system of frequency shifts can be written:
\begin{align}
  \label{eq:fshiftFF}
\mbf{d} \approx -\mbf{V}\hat{\mbf{x}},
\end{align}
where \begin{align*}
\mathbf{d}=\begin{pmatrix}
d_1 \\ \vdots \\ d_N
\end{pmatrix}
\qquad
\mathbf{V}=\begin{pmatrix}
\mathbf{v}_1^T \\ \mathbf{v}_2^T \\ \mathbf{v}_3^T
\end{pmatrix}.
\end{align*}

In practice, the frequency shifts are not observable. Instead the frequency difference of arrival (FDOA) is measured between receivers. The FDOA is equivalent to the difference in frequency shifts,
\begin{align}
  \label{eq:fdoa}
  f_{i,j} = d_j-d_i.
\end{align}
A system equivalent to equation \ref{eq:fshiftFF} can be constructed for the FDOA, with the use of a differencing matrix $\mbf{P}$. The matrix $\mbf{P}$ has entries of 0 and $\pm 1$ corresponding to the differencing in equation \ref{eq:fdoa}. Thus, with the far-field simplification above, the vector of FDOA measurements, $\mbf{f}$, is equivalent to,
\begin{align}
  \label{eq:fdoaFF}
\mbf{f} \approx -\mbf{PV}\hat{\mbf{x}}.
\end{align}
The matrix $\mbf{PV}$ will be referred to as $\tilde{\mbf{V}}$ for simplicity.

This far-field simplification reduces the FDOA equations to a linear system. This suggests that feasible FDOA measurements in the far-field case lie on the image of the unit circle transformed by the matrix $-\tilde{\mbf{V}}$. This image is an ellipse with rotation and scaling determined by the singular value decomposition of $-\tilde{\mbf{V}}$. Indeed, this can be confirmed by computing the singular value decomposition of generated far-field FDOA measurements and confirming they lie on the same subspace as $-\tilde{\mbf{V}}$. This relationship can also be demonstrated visually with a plot of generated FDOA measurements (fig. \ref{f:ellipse}).

\begin{figure}[h!]
  \includegraphics[scale=0.7]{FDOAellipse.png}
  \caption{Plot of far-field $f_{1,2}$ vs. $f_{1,3}$ for a system of three receivers centered around the origin. Note the image is an ellipse with scaling in the direction of the left-singular vectors of $-\tilde{\mbf{V}}$.}
  \label{f:ellipse}
\end{figure}

\red{Include symbolic form of ellipse?}

\subsection{Calculating direction of arrival (DOA)}
The far-field approximated form of the FDOA equations is linear with variable $\hat{\mbf{x}}$, representing the direction of arrival (DOA) of the signal. Thus, the DOA can be found with a linear solve of equation \ref{eq:fdoaFF}. If there are more FDOA measurements than direction components, the pseudo-inverse can be used to solve for $\hat{\mbf{x}}$:
\begin{align}
  \label{eq:doa}
\hat{\mbf{x}} \approx -(\tilde{\mbf{V}}^T\tilde{\mbf{V}})^{-1}\tilde{\mbf{V}}^T\mathbf{f}.
\end{align}
If the receivers move and repeat this process, the intersection of the lines generated by equation \ref{eq:doa} will provide an estimate for the location of the emitter. This is the idea behind our source localization algorithm (section \ref{s:algorithm}).
